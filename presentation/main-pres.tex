\documentclass{beamer}

\usepackage[ngerman]{babel}
\usepackage[utf8]{inputenc}
\usepackage[T1]{fontenc}
\usepackage{lmodern}
\usepackage{color}
\usepackage{enumitem}
\usepackage{amsmath}
\usepackage{array,arydshln} % dashed hlines
\usepackage{CJKutf8}
\usepackage{bxcjkjatype}
\usepackage{tikz}
	\newcommand{\adjustTikzSize}[0]{\Large}
	\newcommand{\tikzScale}[0]{0.75}
\usepackage{amssymb}
\usepackage{graphicx}
	\graphicspath{{img/}}
\usepackage{nicefrac}
\usepackage{csquotes}
\usetheme{Frankfurt}
\usefonttheme{structurebold}
\usecolortheme{rose}
\newtheorem*{bem}{Bemerkung}
\newtheorem*{bsp}{Beispiel}
\usepackage{parskip}
%absolute figure positioning
\usepackage[absolute,overlay]{textpos}
  \setlength{\TPHorizModule}{1mm}
  \setlength{\TPVertModule}{1mm}
\usepackage[backend=bibtex, style=authoryear]{biblatex} 
\addbibresource{literature.bib}

\defbibenvironment{bibliography}
{\list{}
{\setlength{\leftmargin}{\bibhang}%
\setlength{\itemindent}{-\leftmargin}%
\setlength{\itemsep}{6px}%
\setlength{\parsep}{\bibparsep}}}
{\endlist}
{\item \scriptsize}

\definecolor{mygrey}{RGB}{80,80,80}

\setbeamertemplate{headline}
{%
\hfill
\textbf{\insertsection} \
\insertsubsection \
\insertframenumber / \inserttotalframenumber
}
\setbeamertemplate{navigation symbols}{}

\title{Integrating Probabilistic and Database Models for Rapidly Building Customized Machine Learning Applications}
\author{
Frank Rosner
}

\institute{
Martin-Luther-Universität Halle-Wittenberg
}
\date{26. September 2014}

\begin{document}

\frame{\titlepage 
\parbox{0cm}{\tiny 
\vspace{-30pt}\color{mygrey}
\begin{tabbing}
XXXXXXXXXXXXXXXXXXXXXIXXX\=XXXXXXXX\= \kill \\
\>Gutachter:\> Dr. Alexander Hinneburg \\
\>\> Prof. Stefan Brass
\end{tabbing}
}}

\begin{frame}{Test}

x
\begin{definition}[Definition]
y
\end{definition}

\begin{bem}[Bemsss]
x
\end{bem}
\end{frame}

\begin{frame}
\frametitle{Gliederung}
\tableofcontents
\end{frame}

\section{Problemstellung}
\frame{\frametitle{Gliederung} \tableofcontents[currentsection]}
\begin{frame}

\end{frame}

\section{Zielstellung}
\frame{\frametitle{Gliederung} \tableofcontents[currentsection]}
\begin{frame}

\end{frame}

\section{Hauptteil}
\frame{\frametitle{Gliederung} \tableofcontents[currentsection]}
\begin{frame}

\end{frame}

\section{Zusammenfassung}
\frame{\frametitle{Gliederung} \tableofcontents[currentsection]}
\begin{frame}

\end{frame}

\section{Ausblick}
\frame{\frametitle{Gliederung} \tableofcontents[currentsection]}
\begin{frame}

\end{frame}

\end{document}