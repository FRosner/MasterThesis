\documentclass{beamer}

\usepackage[ngerman]{babel}
\usepackage[utf8]{inputenc}
\usepackage[T1]{fontenc}
\usepackage{color}
\usepackage{lmodern}
%\usepackage{enumitem}
%	\setitemize{label=\usebeamerfont*{itemize item}%
%	\usebeamercolor[fg]{itemize item}
%	\usebeamertemplate{itemize item}}
\usepackage{amsmath}
\usepackage{array,arydshln} % dashed hlines
\usepackage{CJKutf8}
\usepackage{bxcjkjatype}
\usepackage{tikz}
	\newcommand{\adjustTikzSize}[0]{\Large}
	\newcommand{\tikzScale}[0]{0.55}
\usepackage{amssymb}
\usepackage{graphicx}
	\graphicspath{{img/}}
	\newcommand{\ptick}[0]{\includegraphics[height=.65em]{tick}}
	\newcommand{\pcross}[0]{\includegraphics[height=.65em]{cross}}
\usepackage{nicefrac}
\usepackage{csquotes}
\usetheme{Frankfurt}
	\usefonttheme{structurebold}
	\usecolortheme{rose}
\newtheorem*{bem}{Bemerkung}
\newtheorem*{bsp}{Beispiel}
\usepackage{parskip}
\RequirePackage{ragged2e} %blocksatz in bloecken
%absolute figure positioning
\usepackage[absolute,overlay]{textpos}
  \setlength{\TPHorizModule}{1mm}
  \setlength{\TPVertModule}{1mm}
\usepackage[backend=bibtex, style=authoryear]{biblatex} 
\addbibresource{literature.bib}

\definecolor{cgray}{gray}{0.35}
\newcommand{\scite}[1]{ {\scriptsize \textcolor{cgray}{(\cite{#1})}} }


  
\defbibenvironment{bibliography}
{\list{}
{\setlength{\itemsep}{6px}%
\setlength{\parsep}{\bibparsep}}}
{\endlist}
{\item \scriptsize}

\definecolor{mygrey}{RGB}{80,80,80}

\setbeamertemplate{headline}
{%
\hfill
\textbf{\insertsection} \
\insertsubsection \
\insertframenumber / \inserttotalframenumber
}
\setbeamertemplate{navigation symbols}{}

\title{Integrating Probabilistic and Database Models for Rapidly Building Customized Machine Learning Applications\vspace*{.4em}}
%Integration von probablistischen und Datenbankmodellen zur schnellen Entwicklung individuell zugeschnittener Machine-Learning-Anwendungen
\author{
\vspace*{-1em}
Frank Rosner
}

\institute{
Martin-Luther-Universität Halle-Wittenberg
}
\date{26. September 2014}

\begin{document}

\frame{\titlepage 
\parbox{0cm}{\tiny 
\vspace{-30pt}\color{mygrey}
\begin{tabbing}
XXXXXXXXXXXXXXXXXXXXXIXXX\=XXXXXXXX\= \kill \\
\>Gutachter:\> Dr. Alexander Hinneburg \\
\>\> Prof. Stefan Brass
\end{tabbing}
}}

\begin{frame}[t]{Motivation}
\vspace*{1.5em}
\begin{itemize}
\item wachsende Datenmassen weltweit
\item Daten als wirtschaftliches Gut\scite{forum2012bigdata}
\end{itemize}

\begin{textblock}{50}(70,70)
	\includegraphics[width=1\linewidth]{electropherogram}
\end{textblock}
\begin{textblock}{25}(10,50)
	\includegraphics[width=1\linewidth]{binary-file}
\end{textblock}
\begin{textblock}{20}(70,45)
	\includegraphics[width=1\linewidth]{globe}
\end{textblock}
\begin{textblock}{10}(90,40)
	\includegraphics[width=1\linewidth]{satellite}
\end{textblock}
\begin{textblock}{10}(100,50)
	\includegraphics[width=1\linewidth]{gps}
\end{textblock}
\begin{textblock}{25}(40,60)
	\includegraphics[width=1\linewidth]{db}
\end{textblock}
\end{frame}

\begin{frame}[t]{Motivation}
\begin{itemize}
\vspace*{1em}
\item \enquote{Data Scientist: Der attraktivste Beruf des 21. Jahrhunderts} \scite{davenport2012data}
	\begin{itemize}
	\item Statistik, Machine-Learning, Mustererkennung, Data Mining
	\item Informatik, effiziente Algorithmen, Datenverwaltung
	\end{itemize}
\end{itemize}

\vspace*{1em}
\textbf{Problem:} Datenflut nicht von Flut an Data Scientists begleitet

\vspace*{3em}
\begin{center}
{\Large Individuell zugeschnittene Analysen}\hspace*{4em}

{\Large ohne Data Scientists?}\hspace*{4em}
\end{center}

\begin{textblock}{30}(93,57)
	\includegraphics[width=1\linewidth]{scientist}
\end{textblock}
\end{frame}

\begin{frame}
\frametitle{Gliederung}
\tableofcontents
\end{frame}

\section{Einleitung}
\frame{\frametitle{Gliederung} \tableofcontents[currentsection]}
\begin{frame}[t]{Problemstellung}
\vspace*{0.7em}
\begin{block}{Rollen in Datenanalyseanwendungen}
	\begin{itemize}
	\setlength{\itemsep}{.4em}
	\item Domänenexperte \hfill (z.B. Japanologe / Soziologe)
	\item Anwendungsentwickler \hfill (z.B. Java + JavaScript)
	\item Data Scientist \hfill (z.B. Dr. Alexander Hinneburg)
	\end{itemize}
\end{block}

\textbf{Probleme}
\begin{itemize}
	\item Eingabedaten hängen sehr stark vom Anwendungsfall ab
	\item gute Data Scientists sind rar und teuer
	\item verfügbare Standardsoftware und -modelle evtl. unzureichend
\end{itemize}

\begin{textblock}{22}(10,77)
	\includegraphics[width=1\linewidth]{weka}
\end{textblock}
\begin{textblock}{22}(40,81)
	\includegraphics[width=1\linewidth]{mallet}
\end{textblock}
\begin{textblock}{22}(70,76)
	\includegraphics[width=1\linewidth]{mlbase}
\end{textblock}
\begin{textblock}{22}(100,81)
	\includegraphics[width=1\linewidth]{scikit}
\end{textblock}
\end{frame}

\begin{frame}{Zielstellung}
\begin{block}{Zielstellung}\justifying
	Entwickeln einer \emph{Datenmodell-getriebenen} Anwendungsarchitektur, welche Anwendungsentwicklern und Domänenexperten erlaubt, \emph{vorhandene Machine-Learning-Algorithmen} individuell für ihr Problem \emph{anzupassen}, ohne einen Data Scientist einstellen zu müssen.
\end{block}

\textbf{Idee}
\begin{itemize}
\item probabilistische Modelle in Form von Entity-Relationship-Modellen (ERMs)
\item Anpassung durch Erweiterung und\\ Integration von ERMs
\item[$\Rightarrow$] kein Wissen über probabilistische Modelle notwendig
\end{itemize}

\begin{textblock}{30}(85,52)
	\scalebox{0.25}{\adjustTikzSize % Graphic for TeX using PGF
% Title: C:\Users\Frank\Documents\1A Uni\1B Master\MasterThesis\presentation\img\MLApplication.dia
% Creator: Dia v0.97.1
% CreationDate: Thu Sep 18 10:58:16 2014
% For: Frank
% \usepackage{tikz}
% The following commands are not supported in PSTricks at present
% We define them conditionally, so when they are implemented,
% this pgf file will use them.
\ifx\du\undefined
  \newlength{\du}
\fi
\setlength{\du}{15\unitlength}
\begin{tikzpicture}
\pgftransformxscale{1.000000}
\pgftransformyscale{-1.000000}
\definecolor{dialinecolor}{rgb}{0.000000, 0.000000, 0.000000}
\pgfsetstrokecolor{dialinecolor}
\definecolor{dialinecolor}{rgb}{1.000000, 1.000000, 1.000000}
\pgfsetfillcolor{dialinecolor}
\pgfsetlinewidth{0.050000\du}
\pgfsetdash{}{0pt}
\pgfsetdash{}{0pt}
\pgfsetbuttcap
{
\definecolor{dialinecolor}{rgb}{0.000000, 0.000000, 0.000000}
\pgfsetfillcolor{dialinecolor}
% was here!!!
\definecolor{dialinecolor}{rgb}{0.000000, 0.000000, 0.000000}
\pgfsetstrokecolor{dialinecolor}
\draw (31.975635\du,8.856631\du)--(30.300000\du,9.440000\du);
}
\pgfsetlinewidth{0.050000\du}
\pgfsetdash{}{0pt}
\pgfsetdash{}{0pt}
\pgfsetbuttcap
{
\definecolor{dialinecolor}{rgb}{0.000000, 0.000000, 0.000000}
\pgfsetfillcolor{dialinecolor}
% was here!!!
\definecolor{dialinecolor}{rgb}{0.000000, 0.000000, 0.000000}
\pgfsetstrokecolor{dialinecolor}
\draw (30.300000\du,9.440000\du)--(28.524023\du,10.112334\du);
}
\pgfsetlinewidth{0.050000\du}
\pgfsetdash{}{0pt}
\pgfsetdash{}{0pt}
\pgfsetmiterjoin
\definecolor{dialinecolor}{rgb}{1.000000, 1.000000, 1.000000}
\pgfsetfillcolor{dialinecolor}
\fill (12.500000\du,10.000000\du)--(12.500000\du,11.000000\du)--(14.500000\du,11.000000\du)--(14.500000\du,10.000000\du)--cycle;
\definecolor{dialinecolor}{rgb}{0.000000, 0.000000, 0.000000}
\pgfsetstrokecolor{dialinecolor}
\draw (12.500000\du,10.000000\du)--(12.500000\du,11.000000\du)--(14.500000\du,11.000000\du)--(14.500000\du,10.000000\du)--cycle;
\pgfsetlinewidth{0.050000\du}
\pgfsetdash{}{0pt}
\pgfsetdash{}{0pt}
\pgfsetmiterjoin
\pgfsetbuttcap
\definecolor{dialinecolor}{rgb}{1.000000, 1.000000, 1.000000}
\pgfsetfillcolor{dialinecolor}
\fill (16.000000\du,10.000000\du)--(16.500000\du,10.500000\du)--(16.000000\du,11.000000\du)--(15.500000\du,10.500000\du)--cycle;
\definecolor{dialinecolor}{rgb}{0.000000, 0.000000, 0.000000}
\pgfsetstrokecolor{dialinecolor}
\draw (16.000000\du,10.000000\du)--(16.500000\du,10.500000\du)--(16.000000\du,11.000000\du)--(15.500000\du,10.500000\du)--cycle;
\pgfsetlinewidth{0.050000\du}
\pgfsetdash{}{0pt}
\pgfsetdash{}{0pt}
\pgfsetmiterjoin
\definecolor{dialinecolor}{rgb}{1.000000, 1.000000, 1.000000}
\pgfsetfillcolor{dialinecolor}
\fill (17.500000\du,10.000000\du)--(17.500000\du,11.000000\du)--(19.500000\du,11.000000\du)--(19.500000\du,10.000000\du)--cycle;
\definecolor{dialinecolor}{rgb}{0.000000, 0.000000, 0.000000}
\pgfsetstrokecolor{dialinecolor}
\draw (17.500000\du,10.000000\du)--(17.500000\du,11.000000\du)--(19.500000\du,11.000000\du)--(19.500000\du,10.000000\du)--cycle;
\pgfsetlinewidth{0.050000\du}
\pgfsetdash{}{0pt}
\pgfsetdash{}{0pt}
\pgfsetbuttcap
{
\definecolor{dialinecolor}{rgb}{0.000000, 0.000000, 0.000000}
\pgfsetfillcolor{dialinecolor}
% was here!!!
\definecolor{dialinecolor}{rgb}{0.000000, 0.000000, 0.000000}
\pgfsetstrokecolor{dialinecolor}
\draw (14.500000\du,10.500000\du)--(15.500000\du,10.500000\du);
}
\pgfsetlinewidth{0.050000\du}
\pgfsetdash{}{0pt}
\pgfsetdash{}{0pt}
\pgfsetbuttcap
{
\definecolor{dialinecolor}{rgb}{0.000000, 0.000000, 0.000000}
\pgfsetfillcolor{dialinecolor}
% was here!!!
\definecolor{dialinecolor}{rgb}{0.000000, 0.000000, 0.000000}
\pgfsetstrokecolor{dialinecolor}
\draw (16.500000\du,10.500000\du)--(17.500000\du,10.500000\du);
}
\definecolor{dialinecolor}{rgb}{1.000000, 1.000000, 1.000000}
\pgfsetfillcolor{dialinecolor}
\pgfpathellipse{\pgfpoint{11.750000\du}{9.250000\du}}{\pgfpoint{0.750000\du}{0\du}}{\pgfpoint{0\du}{0.250000\du}}
\pgfusepath{fill}
\pgfsetlinewidth{0.050000\du}
\pgfsetdash{}{0pt}
\pgfsetdash{}{0pt}
\definecolor{dialinecolor}{rgb}{0.000000, 0.000000, 0.000000}
\pgfsetstrokecolor{dialinecolor}
\pgfpathellipse{\pgfpoint{11.750000\du}{9.250000\du}}{\pgfpoint{0.750000\du}{0\du}}{\pgfpoint{0\du}{0.250000\du}}
\pgfusepath{stroke}
\definecolor{dialinecolor}{rgb}{1.000000, 1.000000, 1.000000}
\pgfsetfillcolor{dialinecolor}
\pgfpathellipse{\pgfpoint{11.250000\du}{10.490000\du}}{\pgfpoint{0.750000\du}{0\du}}{\pgfpoint{0\du}{0.250000\du}}
\pgfusepath{fill}
\pgfsetlinewidth{0.050000\du}
\pgfsetdash{}{0pt}
\pgfsetdash{}{0pt}
\definecolor{dialinecolor}{rgb}{0.000000, 0.000000, 0.000000}
\pgfsetstrokecolor{dialinecolor}
\pgfpathellipse{\pgfpoint{11.250000\du}{10.490000\du}}{\pgfpoint{0.750000\du}{0\du}}{\pgfpoint{0\du}{0.250000\du}}
\pgfusepath{stroke}
\definecolor{dialinecolor}{rgb}{1.000000, 1.000000, 1.000000}
\pgfsetfillcolor{dialinecolor}
\pgfpathellipse{\pgfpoint{11.750000\du}{11.750000\du}}{\pgfpoint{0.750000\du}{0\du}}{\pgfpoint{0\du}{0.250000\du}}
\pgfusepath{fill}
\pgfsetlinewidth{0.050000\du}
\pgfsetdash{}{0pt}
\pgfsetdash{}{0pt}
\definecolor{dialinecolor}{rgb}{0.000000, 0.000000, 0.000000}
\pgfsetstrokecolor{dialinecolor}
\pgfpathellipse{\pgfpoint{11.750000\du}{11.750000\du}}{\pgfpoint{0.750000\du}{0\du}}{\pgfpoint{0\du}{0.250000\du}}
\pgfusepath{stroke}
\definecolor{dialinecolor}{rgb}{1.000000, 1.000000, 1.000000}
\pgfsetfillcolor{dialinecolor}
\pgfpathellipse{\pgfpoint{19.750000\du}{11.750000\du}}{\pgfpoint{0.750000\du}{0\du}}{\pgfpoint{0\du}{0.250000\du}}
\pgfusepath{fill}
\pgfsetlinewidth{0.050000\du}
\pgfsetdash{}{0pt}
\pgfsetdash{}{0pt}
\definecolor{dialinecolor}{rgb}{0.000000, 0.000000, 0.000000}
\pgfsetstrokecolor{dialinecolor}
\pgfpathellipse{\pgfpoint{19.750000\du}{11.750000\du}}{\pgfpoint{0.750000\du}{0\du}}{\pgfpoint{0\du}{0.250000\du}}
\pgfusepath{stroke}
\definecolor{dialinecolor}{rgb}{1.000000, 1.000000, 1.000000}
\pgfsetfillcolor{dialinecolor}
\pgfpathellipse{\pgfpoint{17.250000\du}{11.750000\du}}{\pgfpoint{0.750000\du}{0\du}}{\pgfpoint{0\du}{0.250000\du}}
\pgfusepath{fill}
\pgfsetlinewidth{0.050000\du}
\pgfsetdash{}{0pt}
\pgfsetdash{}{0pt}
\definecolor{dialinecolor}{rgb}{0.000000, 0.000000, 0.000000}
\pgfsetstrokecolor{dialinecolor}
\pgfpathellipse{\pgfpoint{17.250000\du}{11.750000\du}}{\pgfpoint{0.750000\du}{0\du}}{\pgfpoint{0\du}{0.250000\du}}
\pgfusepath{stroke}
\pgfsetlinewidth{0.050000\du}
\pgfsetdash{}{0pt}
\pgfsetdash{}{0pt}
\pgfsetbuttcap
{
\definecolor{dialinecolor}{rgb}{0.000000, 0.000000, 0.000000}
\pgfsetfillcolor{dialinecolor}
% was here!!!
\definecolor{dialinecolor}{rgb}{0.000000, 0.000000, 0.000000}
\pgfsetstrokecolor{dialinecolor}
\draw (12.764709\du,9.974792\du)--(12.087097\du,9.490784\du);
}
\pgfsetlinewidth{0.050000\du}
\pgfsetdash{}{0pt}
\pgfsetdash{}{0pt}
\pgfsetbuttcap
{
\definecolor{dialinecolor}{rgb}{0.000000, 0.000000, 0.000000}
\pgfsetfillcolor{dialinecolor}
% was here!!!
\definecolor{dialinecolor}{rgb}{0.000000, 0.000000, 0.000000}
\pgfsetstrokecolor{dialinecolor}
\draw (12.475525\du,10.495447\du)--(12.024536\du,10.493442\du);
}
\pgfsetlinewidth{0.050000\du}
\pgfsetdash{}{0pt}
\pgfsetdash{}{0pt}
\pgfsetbuttcap
{
\definecolor{dialinecolor}{rgb}{0.000000, 0.000000, 0.000000}
\pgfsetfillcolor{dialinecolor}
% was here!!!
\definecolor{dialinecolor}{rgb}{0.000000, 0.000000, 0.000000}
\pgfsetstrokecolor{dialinecolor}
\draw (12.764709\du,11.025208\du)--(12.087097\du,11.509216\du);
}
\pgfsetlinewidth{0.050000\du}
\pgfsetdash{}{0pt}
\pgfsetdash{}{0pt}
\pgfsetbuttcap
{
\definecolor{dialinecolor}{rgb}{0.000000, 0.000000, 0.000000}
\pgfsetfillcolor{dialinecolor}
% was here!!!
\definecolor{dialinecolor}{rgb}{0.000000, 0.000000, 0.000000}
\pgfsetstrokecolor{dialinecolor}
\draw (19.025208\du,11.025208\du)--(19.495483\du,11.495483\du);
}
\pgfsetlinewidth{0.050000\du}
\pgfsetdash{}{0pt}
\pgfsetdash{}{0pt}
\pgfsetbuttcap
{
\definecolor{dialinecolor}{rgb}{0.000000, 0.000000, 0.000000}
\pgfsetfillcolor{dialinecolor}
% was here!!!
\definecolor{dialinecolor}{rgb}{0.000000, 0.000000, 0.000000}
\pgfsetstrokecolor{dialinecolor}
\draw (17.974792\du,11.025208\du)--(17.504517\du,11.495483\du);
}
\pgfsetlinewidth{0.200000\du}
\pgfsetdash{}{0pt}
\pgfsetdash{}{0pt}
\pgfsetbuttcap
{
\definecolor{dialinecolor}{rgb}{0.823529, 0.823529, 0.823529}
\pgfsetfillcolor{dialinecolor}
% was here!!!
\pgfsetarrowsend{latex}
\definecolor{dialinecolor}{rgb}{0.823529, 0.823529, 0.823529}
\pgfsetstrokecolor{dialinecolor}
\draw (21.300000\du,12.300000\du)--(22.500000\du,13.350000\du);
}
\pgfsetlinewidth{0.200000\du}
\pgfsetdash{}{0pt}
\pgfsetdash{}{0pt}
\pgfsetbuttcap
{
\definecolor{dialinecolor}{rgb}{0.823529, 0.823529, 0.823529}
\pgfsetfillcolor{dialinecolor}
% was here!!!
\pgfsetarrowsend{latex}
\definecolor{dialinecolor}{rgb}{0.823529, 0.823529, 0.823529}
\pgfsetstrokecolor{dialinecolor}
\draw (25.700000\du,12.300000\du)--(24.500000\du,13.350000\du);
}
\pgfsetlinewidth{0.050000\du}
\pgfsetdash{}{0pt}
\pgfsetdash{}{0pt}
\pgfsetmiterjoin
\pgfsetbuttcap
\definecolor{dialinecolor}{rgb}{1.000000, 1.000000, 1.000000}
\pgfsetfillcolor{dialinecolor}
\fill (18.500000\du,8.000000\du)--(19.000000\du,8.500000\du)--(18.500000\du,9.000000\du)--(18.000000\du,8.500000\du)--cycle;
\definecolor{dialinecolor}{rgb}{0.000000, 0.000000, 0.000000}
\pgfsetstrokecolor{dialinecolor}
\draw (18.500000\du,8.000000\du)--(19.000000\du,8.500000\du)--(18.500000\du,9.000000\du)--(18.000000\du,8.500000\du)--cycle;
\pgfsetlinewidth{0.050000\du}
\pgfsetdash{}{0pt}
\pgfsetdash{}{0pt}
\pgfsetbuttcap
{
\definecolor{dialinecolor}{rgb}{0.000000, 0.000000, 0.000000}
\pgfsetfillcolor{dialinecolor}
% was here!!!
\definecolor{dialinecolor}{rgb}{0.000000, 0.000000, 0.000000}
\pgfsetstrokecolor{dialinecolor}
\draw (18.500000\du,10.000000\du)--(18.500000\du,9.000000\du);
}
\pgfsetlinewidth{0.050000\du}
\pgfsetdash{}{0pt}
\pgfsetdash{}{0pt}
\pgfsetbuttcap
{
\definecolor{dialinecolor}{rgb}{0.000000, 0.000000, 0.000000}
\pgfsetfillcolor{dialinecolor}
% was here!!!
\definecolor{dialinecolor}{rgb}{0.000000, 0.000000, 0.000000}
\pgfsetstrokecolor{dialinecolor}
\draw (18.500000\du,7.000000\du)--(18.500000\du,8.000000\du);
}
\pgfsetlinewidth{0.050000\du}
\pgfsetdash{}{0pt}
\pgfsetdash{}{0pt}
\pgfsetmiterjoin
\definecolor{dialinecolor}{rgb}{1.000000, 1.000000, 1.000000}
\pgfsetfillcolor{dialinecolor}
\fill (17.500000\du,6.000000\du)--(17.500000\du,7.000000\du)--(19.500000\du,7.000000\du)--(19.500000\du,6.000000\du)--cycle;
\definecolor{dialinecolor}{rgb}{0.000000, 0.000000, 0.000000}
\pgfsetstrokecolor{dialinecolor}
\draw (17.500000\du,6.000000\du)--(17.500000\du,7.000000\du)--(19.500000\du,7.000000\du)--(19.500000\du,6.000000\du)--cycle;
\definecolor{dialinecolor}{rgb}{1.000000, 1.000000, 1.000000}
\pgfsetfillcolor{dialinecolor}
\pgfpathellipse{\pgfpoint{17.250000\du}{5.250000\du}}{\pgfpoint{0.750000\du}{0\du}}{\pgfpoint{0\du}{0.250000\du}}
\pgfusepath{fill}
\pgfsetlinewidth{0.050000\du}
\pgfsetdash{}{0pt}
\pgfsetdash{}{0pt}
\definecolor{dialinecolor}{rgb}{0.000000, 0.000000, 0.000000}
\pgfsetstrokecolor{dialinecolor}
\pgfpathellipse{\pgfpoint{17.250000\du}{5.250000\du}}{\pgfpoint{0.750000\du}{0\du}}{\pgfpoint{0\du}{0.250000\du}}
\pgfusepath{stroke}
\pgfsetlinewidth{0.050000\du}
\pgfsetdash{}{0pt}
\pgfsetdash{}{0pt}
\pgfsetbuttcap
{
\definecolor{dialinecolor}{rgb}{0.000000, 0.000000, 0.000000}
\pgfsetfillcolor{dialinecolor}
% was here!!!
\definecolor{dialinecolor}{rgb}{0.000000, 0.000000, 0.000000}
\pgfsetstrokecolor{dialinecolor}
\draw (17.974792\du,5.974792\du)--(17.504517\du,5.504517\du);
}
\pgfsetlinewidth{0.050000\du}
\pgfsetdash{}{0pt}
\pgfsetdash{}{0pt}
\pgfsetmiterjoin
\definecolor{dialinecolor}{rgb}{1.000000, 1.000000, 1.000000}
\pgfsetfillcolor{dialinecolor}
\fill (32.000000\du,8.000000\du)--(32.000000\du,9.000000\du)--(34.000000\du,9.000000\du)--(34.000000\du,8.000000\du)--cycle;
\definecolor{dialinecolor}{rgb}{0.000000, 0.000000, 0.000000}
\pgfsetstrokecolor{dialinecolor}
\draw (32.000000\du,8.000000\du)--(32.000000\du,9.000000\du)--(34.000000\du,9.000000\du)--(34.000000\du,8.000000\du)--cycle;
\pgfsetlinewidth{0.050000\du}
\pgfsetdash{}{0pt}
\pgfsetdash{}{0pt}
\pgfsetmiterjoin
\pgfsetbuttcap
\definecolor{dialinecolor}{rgb}{1.000000, 1.000000, 1.000000}
\pgfsetfillcolor{dialinecolor}
\fill (30.300000\du,8.940000\du)--(30.800000\du,9.440000\du)--(30.300000\du,9.940000\du)--(29.800000\du,9.440000\du)--cycle;
\definecolor{dialinecolor}{rgb}{0.000000, 0.000000, 0.000000}
\pgfsetstrokecolor{dialinecolor}
\draw (30.300000\du,8.940000\du)--(30.800000\du,9.440000\du)--(30.300000\du,9.940000\du)--(29.800000\du,9.440000\du)--cycle;
\pgfsetlinewidth{0.050000\du}
\pgfsetdash{}{0pt}
\pgfsetdash{}{0pt}
\pgfsetmiterjoin
\definecolor{dialinecolor}{rgb}{1.000000, 1.000000, 1.000000}
\pgfsetfillcolor{dialinecolor}
\fill (26.500000\du,10.000000\du)--(26.500000\du,11.000000\du)--(28.500000\du,11.000000\du)--(28.500000\du,10.000000\du)--cycle;
\definecolor{dialinecolor}{rgb}{0.000000, 0.000000, 0.000000}
\pgfsetstrokecolor{dialinecolor}
\draw (26.500000\du,10.000000\du)--(26.500000\du,11.000000\du)--(28.500000\du,11.000000\du)--(28.500000\du,10.000000\du)--cycle;
\definecolor{dialinecolor}{rgb}{1.000000, 1.000000, 1.000000}
\pgfsetfillcolor{dialinecolor}
\pgfpathellipse{\pgfpoint{35.750000\du}{7.750000\du}}{\pgfpoint{0.750000\du}{0\du}}{\pgfpoint{0\du}{0.250000\du}}
\pgfusepath{fill}
\pgfsetlinewidth{0.050000\du}
\pgfsetdash{}{0pt}
\pgfsetdash{}{0pt}
\definecolor{dialinecolor}{rgb}{0.000000, 0.000000, 0.000000}
\pgfsetstrokecolor{dialinecolor}
\pgfpathellipse{\pgfpoint{35.750000\du}{7.750000\du}}{\pgfpoint{0.750000\du}{0\du}}{\pgfpoint{0\du}{0.250000\du}}
\pgfusepath{stroke}
\definecolor{dialinecolor}{rgb}{1.000000, 1.000000, 1.000000}
\pgfsetfillcolor{dialinecolor}
\pgfpathellipse{\pgfpoint{35.750000\du}{9.250000\du}}{\pgfpoint{0.750000\du}{0\du}}{\pgfpoint{0\du}{0.250000\du}}
\pgfusepath{fill}
\pgfsetlinewidth{0.050000\du}
\pgfsetdash{}{0pt}
\pgfsetdash{}{0pt}
\definecolor{dialinecolor}{rgb}{0.000000, 0.000000, 0.000000}
\pgfsetstrokecolor{dialinecolor}
\pgfpathellipse{\pgfpoint{35.750000\du}{9.250000\du}}{\pgfpoint{0.750000\du}{0\du}}{\pgfpoint{0\du}{0.250000\du}}
\pgfusepath{stroke}
\pgfsetlinewidth{0.050000\du}
\pgfsetdash{}{0pt}
\pgfsetdash{}{0pt}
\pgfsetbuttcap
{
\definecolor{dialinecolor}{rgb}{0.000000, 0.000000, 0.000000}
\pgfsetfillcolor{dialinecolor}
% was here!!!
\definecolor{dialinecolor}{rgb}{0.000000, 0.000000, 0.000000}
\pgfsetstrokecolor{dialinecolor}
\draw (34.023193\du,8.220947\du)--(35.145416\du,7.914886\du);
}
\pgfsetlinewidth{0.050000\du}
\pgfsetdash{}{0pt}
\pgfsetdash{}{0pt}
\pgfsetbuttcap
{
\definecolor{dialinecolor}{rgb}{0.000000, 0.000000, 0.000000}
\pgfsetfillcolor{dialinecolor}
% was here!!!
\definecolor{dialinecolor}{rgb}{0.000000, 0.000000, 0.000000}
\pgfsetstrokecolor{dialinecolor}
\draw (34.023193\du,8.779053\du)--(35.145416\du,9.085114\du);
}
\pgfsetlinewidth{0.050000\du}
\pgfsetdash{}{0pt}
\pgfsetdash{}{0pt}
\pgfsetmiterjoin
\pgfsetbuttcap
\definecolor{dialinecolor}{rgb}{1.000000, 1.000000, 1.000000}
\pgfsetfillcolor{dialinecolor}
\fill (27.500000\du,8.000000\du)--(28.000000\du,8.500000\du)--(27.500000\du,9.000000\du)--(27.000000\du,8.500000\du)--cycle;
\definecolor{dialinecolor}{rgb}{0.000000, 0.000000, 0.000000}
\pgfsetstrokecolor{dialinecolor}
\draw (27.500000\du,8.000000\du)--(28.000000\du,8.500000\du)--(27.500000\du,9.000000\du)--(27.000000\du,8.500000\du)--cycle;
\pgfsetlinewidth{0.050000\du}
\pgfsetdash{}{0pt}
\pgfsetdash{}{0pt}
\pgfsetbuttcap
{
\definecolor{dialinecolor}{rgb}{0.000000, 0.000000, 0.000000}
\pgfsetfillcolor{dialinecolor}
% was here!!!
\definecolor{dialinecolor}{rgb}{0.000000, 0.000000, 0.000000}
\pgfsetstrokecolor{dialinecolor}
\draw (27.500000\du,10.000000\du)--(27.500000\du,9.000000\du);
}
\pgfsetlinewidth{0.050000\du}
\pgfsetdash{}{0pt}
\pgfsetdash{}{0pt}
\pgfsetbuttcap
{
\definecolor{dialinecolor}{rgb}{0.000000, 0.000000, 0.000000}
\pgfsetfillcolor{dialinecolor}
% was here!!!
\definecolor{dialinecolor}{rgb}{0.000000, 0.000000, 0.000000}
\pgfsetstrokecolor{dialinecolor}
\draw (27.500000\du,7.000000\du)--(27.500000\du,8.000000\du);
}
\pgfsetlinewidth{0.050000\du}
\pgfsetdash{}{0pt}
\pgfsetdash{}{0pt}
\pgfsetmiterjoin
\definecolor{dialinecolor}{rgb}{1.000000, 1.000000, 1.000000}
\pgfsetfillcolor{dialinecolor}
\fill (26.500000\du,6.000000\du)--(26.500000\du,7.000000\du)--(28.500000\du,7.000000\du)--(28.500000\du,6.000000\du)--cycle;
\definecolor{dialinecolor}{rgb}{0.000000, 0.000000, 0.000000}
\pgfsetstrokecolor{dialinecolor}
\draw (26.500000\du,6.000000\du)--(26.500000\du,7.000000\du)--(28.500000\du,7.000000\du)--(28.500000\du,6.000000\du)--cycle;
\definecolor{dialinecolor}{rgb}{1.000000, 1.000000, 1.000000}
\pgfsetfillcolor{dialinecolor}
\pgfpathellipse{\pgfpoint{28.750000\du}{5.250000\du}}{\pgfpoint{0.750000\du}{0\du}}{\pgfpoint{0\du}{0.250000\du}}
\pgfusepath{fill}
\pgfsetlinewidth{0.050000\du}
\pgfsetdash{}{0pt}
\pgfsetdash{}{0pt}
\definecolor{dialinecolor}{rgb}{0.000000, 0.000000, 0.000000}
\pgfsetstrokecolor{dialinecolor}
\pgfpathellipse{\pgfpoint{28.750000\du}{5.250000\du}}{\pgfpoint{0.750000\du}{0\du}}{\pgfpoint{0\du}{0.250000\du}}
\pgfusepath{stroke}
\pgfsetlinewidth{0.050000\du}
\pgfsetdash{}{0pt}
\pgfsetdash{}{0pt}
\pgfsetbuttcap
{
\definecolor{dialinecolor}{rgb}{0.000000, 0.000000, 0.000000}
\pgfsetfillcolor{dialinecolor}
% was here!!!
\definecolor{dialinecolor}{rgb}{0.000000, 0.000000, 0.000000}
\pgfsetstrokecolor{dialinecolor}
\draw (28.025208\du,5.974792\du)--(28.495483\du,5.504517\du);
}
\pgfsetlinewidth{0.050000\du}
\pgfsetdash{}{0pt}
\pgfsetdash{}{0pt}
\pgfsetbuttcap
{
\definecolor{dialinecolor}{rgb}{0.000000, 0.000000, 0.000000}
\pgfsetfillcolor{dialinecolor}
% was here!!!
\definecolor{dialinecolor}{rgb}{0.000000, 0.000000, 0.000000}
\pgfsetstrokecolor{dialinecolor}
\draw (31.975635\du,8.143369\du)--(30.300000\du,7.560000\du);
}
\pgfsetlinewidth{0.050000\du}
\pgfsetdash{}{0pt}
\pgfsetdash{}{0pt}
\pgfsetbuttcap
{
\definecolor{dialinecolor}{rgb}{0.000000, 0.000000, 0.000000}
\pgfsetfillcolor{dialinecolor}
% was here!!!
\definecolor{dialinecolor}{rgb}{0.000000, 0.000000, 0.000000}
\pgfsetstrokecolor{dialinecolor}
\draw (30.300000\du,7.560000\du)--(28.524023\du,6.887666\du);
}
\pgfsetlinewidth{0.050000\du}
\pgfsetdash{}{0pt}
\pgfsetdash{}{0pt}
\pgfsetmiterjoin
\pgfsetbuttcap
\definecolor{dialinecolor}{rgb}{1.000000, 1.000000, 1.000000}
\pgfsetfillcolor{dialinecolor}
\fill (30.300000\du,7.060000\du)--(30.800000\du,7.560000\du)--(30.300000\du,8.060000\du)--(29.800000\du,7.560000\du)--cycle;
\definecolor{dialinecolor}{rgb}{0.000000, 0.000000, 0.000000}
\pgfsetstrokecolor{dialinecolor}
\draw (30.300000\du,7.060000\du)--(30.800000\du,7.560000\du)--(30.300000\du,8.060000\du)--(29.800000\du,7.560000\du)--cycle;
\pgfsetlinewidth{0.050000\du}
\pgfsetdash{}{0pt}
\pgfsetdash{}{0pt}
\pgfsetmiterjoin
\definecolor{dialinecolor}{rgb}{1.000000, 1.000000, 1.000000}
\pgfsetfillcolor{dialinecolor}
\fill (17.500000\du,19.000000\du)--(17.500000\du,20.000000\du)--(19.500000\du,20.000000\du)--(19.500000\du,19.000000\du)--cycle;
\definecolor{dialinecolor}{rgb}{0.000000, 0.000000, 0.000000}
\pgfsetstrokecolor{dialinecolor}
\draw (17.500000\du,19.000000\du)--(17.500000\du,20.000000\du)--(19.500000\du,20.000000\du)--(19.500000\du,19.000000\du)--cycle;
\pgfsetlinewidth{0.050000\du}
\pgfsetdash{}{0pt}
\pgfsetdash{}{0pt}
\pgfsetmiterjoin
\pgfsetbuttcap
\definecolor{dialinecolor}{rgb}{1.000000, 1.000000, 1.000000}
\pgfsetfillcolor{dialinecolor}
\fill (21.000000\du,19.000000\du)--(21.500000\du,19.500000\du)--(21.000000\du,20.000000\du)--(20.500000\du,19.500000\du)--cycle;
\definecolor{dialinecolor}{rgb}{0.000000, 0.000000, 0.000000}
\pgfsetstrokecolor{dialinecolor}
\draw (21.000000\du,19.000000\du)--(21.500000\du,19.500000\du)--(21.000000\du,20.000000\du)--(20.500000\du,19.500000\du)--cycle;
\pgfsetlinewidth{0.050000\du}
\pgfsetdash{}{0pt}
\pgfsetdash{}{0pt}
\pgfsetmiterjoin
\definecolor{dialinecolor}{rgb}{1.000000, 1.000000, 1.000000}
\pgfsetfillcolor{dialinecolor}
\fill (22.500000\du,19.000000\du)--(22.500000\du,20.000000\du)--(24.500000\du,20.000000\du)--(24.500000\du,19.000000\du)--cycle;
\definecolor{dialinecolor}{rgb}{0.000000, 0.000000, 0.000000}
\pgfsetstrokecolor{dialinecolor}
\draw (22.500000\du,19.000000\du)--(22.500000\du,20.000000\du)--(24.500000\du,20.000000\du)--(24.500000\du,19.000000\du)--cycle;
\pgfsetlinewidth{0.050000\du}
\pgfsetdash{}{0pt}
\pgfsetdash{}{0pt}
\pgfsetbuttcap
{
\definecolor{dialinecolor}{rgb}{0.000000, 0.000000, 0.000000}
\pgfsetfillcolor{dialinecolor}
% was here!!!
\definecolor{dialinecolor}{rgb}{0.000000, 0.000000, 0.000000}
\pgfsetstrokecolor{dialinecolor}
\draw (19.500000\du,19.500000\du)--(20.500000\du,19.500000\du);
}
\pgfsetlinewidth{0.050000\du}
\pgfsetdash{}{0pt}
\pgfsetdash{}{0pt}
\pgfsetbuttcap
{
\definecolor{dialinecolor}{rgb}{0.000000, 0.000000, 0.000000}
\pgfsetfillcolor{dialinecolor}
% was here!!!
\definecolor{dialinecolor}{rgb}{0.000000, 0.000000, 0.000000}
\pgfsetstrokecolor{dialinecolor}
\draw (21.500000\du,19.500000\du)--(22.500000\du,19.500000\du);
}
\definecolor{dialinecolor}{rgb}{1.000000, 1.000000, 1.000000}
\pgfsetfillcolor{dialinecolor}
\pgfpathellipse{\pgfpoint{16.750000\du}{18.250000\du}}{\pgfpoint{0.750000\du}{0\du}}{\pgfpoint{0\du}{0.250000\du}}
\pgfusepath{fill}
\pgfsetlinewidth{0.050000\du}
\pgfsetdash{}{0pt}
\pgfsetdash{}{0pt}
\definecolor{dialinecolor}{rgb}{0.000000, 0.000000, 0.000000}
\pgfsetstrokecolor{dialinecolor}
\pgfpathellipse{\pgfpoint{16.750000\du}{18.250000\du}}{\pgfpoint{0.750000\du}{0\du}}{\pgfpoint{0\du}{0.250000\du}}
\pgfusepath{stroke}
\definecolor{dialinecolor}{rgb}{1.000000, 1.000000, 1.000000}
\pgfsetfillcolor{dialinecolor}
\pgfpathellipse{\pgfpoint{16.250000\du}{19.490000\du}}{\pgfpoint{0.750000\du}{0\du}}{\pgfpoint{0\du}{0.250000\du}}
\pgfusepath{fill}
\pgfsetlinewidth{0.050000\du}
\pgfsetdash{}{0pt}
\pgfsetdash{}{0pt}
\definecolor{dialinecolor}{rgb}{0.000000, 0.000000, 0.000000}
\pgfsetstrokecolor{dialinecolor}
\pgfpathellipse{\pgfpoint{16.250000\du}{19.490000\du}}{\pgfpoint{0.750000\du}{0\du}}{\pgfpoint{0\du}{0.250000\du}}
\pgfusepath{stroke}
\definecolor{dialinecolor}{rgb}{1.000000, 1.000000, 1.000000}
\pgfsetfillcolor{dialinecolor}
\pgfpathellipse{\pgfpoint{16.750000\du}{20.750000\du}}{\pgfpoint{0.750000\du}{0\du}}{\pgfpoint{0\du}{0.250000\du}}
\pgfusepath{fill}
\pgfsetlinewidth{0.050000\du}
\pgfsetdash{}{0pt}
\pgfsetdash{}{0pt}
\definecolor{dialinecolor}{rgb}{0.000000, 0.000000, 0.000000}
\pgfsetstrokecolor{dialinecolor}
\pgfpathellipse{\pgfpoint{16.750000\du}{20.750000\du}}{\pgfpoint{0.750000\du}{0\du}}{\pgfpoint{0\du}{0.250000\du}}
\pgfusepath{stroke}
\definecolor{dialinecolor}{rgb}{1.000000, 1.000000, 1.000000}
\pgfsetfillcolor{dialinecolor}
\pgfpathellipse{\pgfpoint{24.750000\du}{20.750000\du}}{\pgfpoint{0.750000\du}{0\du}}{\pgfpoint{0\du}{0.250000\du}}
\pgfusepath{fill}
\pgfsetlinewidth{0.050000\du}
\pgfsetdash{}{0pt}
\pgfsetdash{}{0pt}
\definecolor{dialinecolor}{rgb}{0.000000, 0.000000, 0.000000}
\pgfsetstrokecolor{dialinecolor}
\pgfpathellipse{\pgfpoint{24.750000\du}{20.750000\du}}{\pgfpoint{0.750000\du}{0\du}}{\pgfpoint{0\du}{0.250000\du}}
\pgfusepath{stroke}
\definecolor{dialinecolor}{rgb}{1.000000, 1.000000, 1.000000}
\pgfsetfillcolor{dialinecolor}
\pgfpathellipse{\pgfpoint{22.250000\du}{20.750000\du}}{\pgfpoint{0.750000\du}{0\du}}{\pgfpoint{0\du}{0.250000\du}}
\pgfusepath{fill}
\pgfsetlinewidth{0.050000\du}
\pgfsetdash{}{0pt}
\pgfsetdash{}{0pt}
\definecolor{dialinecolor}{rgb}{0.000000, 0.000000, 0.000000}
\pgfsetstrokecolor{dialinecolor}
\pgfpathellipse{\pgfpoint{22.250000\du}{20.750000\du}}{\pgfpoint{0.750000\du}{0\du}}{\pgfpoint{0\du}{0.250000\du}}
\pgfusepath{stroke}
\pgfsetlinewidth{0.050000\du}
\pgfsetdash{}{0pt}
\pgfsetdash{}{0pt}
\pgfsetbuttcap
{
\definecolor{dialinecolor}{rgb}{0.000000, 0.000000, 0.000000}
\pgfsetfillcolor{dialinecolor}
% was here!!!
\definecolor{dialinecolor}{rgb}{0.000000, 0.000000, 0.000000}
\pgfsetstrokecolor{dialinecolor}
\draw (17.764709\du,18.974792\du)--(17.087097\du,18.490784\du);
}
\pgfsetlinewidth{0.050000\du}
\pgfsetdash{}{0pt}
\pgfsetdash{}{0pt}
\pgfsetbuttcap
{
\definecolor{dialinecolor}{rgb}{0.000000, 0.000000, 0.000000}
\pgfsetfillcolor{dialinecolor}
% was here!!!
\definecolor{dialinecolor}{rgb}{0.000000, 0.000000, 0.000000}
\pgfsetstrokecolor{dialinecolor}
\draw (17.475525\du,19.495447\du)--(17.024536\du,19.493442\du);
}
\pgfsetlinewidth{0.050000\du}
\pgfsetdash{}{0pt}
\pgfsetdash{}{0pt}
\pgfsetbuttcap
{
\definecolor{dialinecolor}{rgb}{0.000000, 0.000000, 0.000000}
\pgfsetfillcolor{dialinecolor}
% was here!!!
\definecolor{dialinecolor}{rgb}{0.000000, 0.000000, 0.000000}
\pgfsetstrokecolor{dialinecolor}
\draw (17.764709\du,20.025208\du)--(17.087097\du,20.509216\du);
}
\pgfsetlinewidth{0.050000\du}
\pgfsetdash{}{0pt}
\pgfsetdash{}{0pt}
\pgfsetbuttcap
{
\definecolor{dialinecolor}{rgb}{0.000000, 0.000000, 0.000000}
\pgfsetfillcolor{dialinecolor}
% was here!!!
\definecolor{dialinecolor}{rgb}{0.000000, 0.000000, 0.000000}
\pgfsetstrokecolor{dialinecolor}
\draw (24.025208\du,20.025208\du)--(24.495483\du,20.495483\du);
}
\pgfsetlinewidth{0.050000\du}
\pgfsetdash{}{0pt}
\pgfsetdash{}{0pt}
\pgfsetbuttcap
{
\definecolor{dialinecolor}{rgb}{0.000000, 0.000000, 0.000000}
\pgfsetfillcolor{dialinecolor}
% was here!!!
\definecolor{dialinecolor}{rgb}{0.000000, 0.000000, 0.000000}
\pgfsetstrokecolor{dialinecolor}
\draw (22.974792\du,20.025208\du)--(22.504517\du,20.495483\du);
}
\pgfsetlinewidth{0.050000\du}
\pgfsetdash{}{0pt}
\pgfsetdash{}{0pt}
\pgfsetmiterjoin
\pgfsetbuttcap
\definecolor{dialinecolor}{rgb}{1.000000, 1.000000, 1.000000}
\pgfsetfillcolor{dialinecolor}
\fill (23.500000\du,17.000000\du)--(24.000000\du,17.500000\du)--(23.500000\du,18.000000\du)--(23.000000\du,17.500000\du)--cycle;
\definecolor{dialinecolor}{rgb}{0.000000, 0.000000, 0.000000}
\pgfsetstrokecolor{dialinecolor}
\draw (23.500000\du,17.000000\du)--(24.000000\du,17.500000\du)--(23.500000\du,18.000000\du)--(23.000000\du,17.500000\du)--cycle;
\pgfsetlinewidth{0.050000\du}
\pgfsetdash{}{0pt}
\pgfsetdash{}{0pt}
\pgfsetbuttcap
{
\definecolor{dialinecolor}{rgb}{0.000000, 0.000000, 0.000000}
\pgfsetfillcolor{dialinecolor}
% was here!!!
\definecolor{dialinecolor}{rgb}{0.000000, 0.000000, 0.000000}
\pgfsetstrokecolor{dialinecolor}
\draw (23.500000\du,19.000000\du)--(23.500000\du,18.000000\du);
}
\pgfsetlinewidth{0.050000\du}
\pgfsetdash{}{0pt}
\pgfsetdash{}{0pt}
\pgfsetbuttcap
{
\definecolor{dialinecolor}{rgb}{0.000000, 0.000000, 0.000000}
\pgfsetfillcolor{dialinecolor}
% was here!!!
\definecolor{dialinecolor}{rgb}{0.000000, 0.000000, 0.000000}
\pgfsetstrokecolor{dialinecolor}
\draw (23.500000\du,16.000000\du)--(23.500000\du,17.000000\du);
}
\pgfsetlinewidth{0.050000\du}
\pgfsetdash{}{0pt}
\pgfsetdash{}{0pt}
\pgfsetmiterjoin
\definecolor{dialinecolor}{rgb}{1.000000, 1.000000, 1.000000}
\pgfsetfillcolor{dialinecolor}
\fill (22.500000\du,15.000000\du)--(22.500000\du,16.000000\du)--(24.500000\du,16.000000\du)--(24.500000\du,15.000000\du)--cycle;
\definecolor{dialinecolor}{rgb}{0.000000, 0.000000, 0.000000}
\pgfsetstrokecolor{dialinecolor}
\draw (22.500000\du,15.000000\du)--(22.500000\du,16.000000\du)--(24.500000\du,16.000000\du)--(24.500000\du,15.000000\du)--cycle;
\definecolor{dialinecolor}{rgb}{1.000000, 1.000000, 1.000000}
\pgfsetfillcolor{dialinecolor}
\pgfpathellipse{\pgfpoint{22.250000\du}{14.250000\du}}{\pgfpoint{0.750000\du}{0\du}}{\pgfpoint{0\du}{0.250000\du}}
\pgfusepath{fill}
\pgfsetlinewidth{0.050000\du}
\pgfsetdash{}{0pt}
\pgfsetdash{}{0pt}
\definecolor{dialinecolor}{rgb}{0.000000, 0.000000, 0.000000}
\pgfsetstrokecolor{dialinecolor}
\pgfpathellipse{\pgfpoint{22.250000\du}{14.250000\du}}{\pgfpoint{0.750000\du}{0\du}}{\pgfpoint{0\du}{0.250000\du}}
\pgfusepath{stroke}
\pgfsetlinewidth{0.050000\du}
\pgfsetdash{}{0pt}
\pgfsetdash{}{0pt}
\pgfsetbuttcap
{
\definecolor{dialinecolor}{rgb}{0.000000, 0.000000, 0.000000}
\pgfsetfillcolor{dialinecolor}
% was here!!!
\definecolor{dialinecolor}{rgb}{0.000000, 0.000000, 0.000000}
\pgfsetstrokecolor{dialinecolor}
\draw (22.974792\du,14.974792\du)--(22.504517\du,14.504517\du);
}
\definecolor{dialinecolor}{rgb}{1.000000, 1.000000, 1.000000}
\pgfsetfillcolor{dialinecolor}
\pgfpathellipse{\pgfpoint{24.750000\du}{14.250000\du}}{\pgfpoint{0.750000\du}{0\du}}{\pgfpoint{0\du}{0.250000\du}}
\pgfusepath{fill}
\pgfsetlinewidth{0.050000\du}
\pgfsetdash{}{0pt}
\pgfsetdash{}{0pt}
\definecolor{dialinecolor}{rgb}{0.000000, 0.000000, 0.000000}
\pgfsetstrokecolor{dialinecolor}
\pgfpathellipse{\pgfpoint{24.750000\du}{14.250000\du}}{\pgfpoint{0.750000\du}{0\du}}{\pgfpoint{0\du}{0.250000\du}}
\pgfusepath{stroke}
\pgfsetlinewidth{0.050000\du}
\pgfsetdash{}{0pt}
\pgfsetdash{}{0pt}
\pgfsetbuttcap
{
\definecolor{dialinecolor}{rgb}{0.000000, 0.000000, 0.000000}
\pgfsetfillcolor{dialinecolor}
% was here!!!
\definecolor{dialinecolor}{rgb}{0.000000, 0.000000, 0.000000}
\pgfsetstrokecolor{dialinecolor}
\draw (24.025457\du,14.974543\du)--(24.495500\du,14.504500\du);
}
\pgfsetlinewidth{0.050000\du}
\pgfsetdash{}{0pt}
\pgfsetdash{}{0pt}
\pgfsetbuttcap
{
\definecolor{dialinecolor}{rgb}{0.000000, 0.000000, 0.000000}
\pgfsetfillcolor{dialinecolor}
% was here!!!
\definecolor{dialinecolor}{rgb}{0.000000, 0.000000, 0.000000}
\pgfsetstrokecolor{dialinecolor}
\draw (27.975635\du,17.856631\du)--(26.300000\du,18.440000\du);
}
\pgfsetlinewidth{0.050000\du}
\pgfsetdash{}{0pt}
\pgfsetdash{}{0pt}
\pgfsetbuttcap
{
\definecolor{dialinecolor}{rgb}{0.000000, 0.000000, 0.000000}
\pgfsetfillcolor{dialinecolor}
% was here!!!
\definecolor{dialinecolor}{rgb}{0.000000, 0.000000, 0.000000}
\pgfsetstrokecolor{dialinecolor}
\draw (26.300000\du,18.440000\du)--(24.524023\du,19.112334\du);
}
\pgfsetlinewidth{0.050000\du}
\pgfsetdash{}{0pt}
\pgfsetdash{}{0pt}
\pgfsetmiterjoin
\definecolor{dialinecolor}{rgb}{1.000000, 1.000000, 1.000000}
\pgfsetfillcolor{dialinecolor}
\fill (28.000000\du,17.000000\du)--(28.000000\du,18.000000\du)--(30.000000\du,18.000000\du)--(30.000000\du,17.000000\du)--cycle;
\definecolor{dialinecolor}{rgb}{0.000000, 0.000000, 0.000000}
\pgfsetstrokecolor{dialinecolor}
\draw (28.000000\du,17.000000\du)--(28.000000\du,18.000000\du)--(30.000000\du,18.000000\du)--(30.000000\du,17.000000\du)--cycle;
\pgfsetlinewidth{0.050000\du}
\pgfsetdash{}{0pt}
\pgfsetdash{}{0pt}
\pgfsetmiterjoin
\pgfsetbuttcap
\definecolor{dialinecolor}{rgb}{1.000000, 1.000000, 1.000000}
\pgfsetfillcolor{dialinecolor}
\fill (26.300000\du,17.940000\du)--(26.800000\du,18.440000\du)--(26.300000\du,18.940000\du)--(25.800000\du,18.440000\du)--cycle;
\definecolor{dialinecolor}{rgb}{0.000000, 0.000000, 0.000000}
\pgfsetstrokecolor{dialinecolor}
\draw (26.300000\du,17.940000\du)--(26.800000\du,18.440000\du)--(26.300000\du,18.940000\du)--(25.800000\du,18.440000\du)--cycle;
\definecolor{dialinecolor}{rgb}{1.000000, 1.000000, 1.000000}
\pgfsetfillcolor{dialinecolor}
\pgfpathellipse{\pgfpoint{31.750000\du}{16.750000\du}}{\pgfpoint{0.750000\du}{0\du}}{\pgfpoint{0\du}{0.250000\du}}
\pgfusepath{fill}
\pgfsetlinewidth{0.050000\du}
\pgfsetdash{}{0pt}
\pgfsetdash{}{0pt}
\definecolor{dialinecolor}{rgb}{0.000000, 0.000000, 0.000000}
\pgfsetstrokecolor{dialinecolor}
\pgfpathellipse{\pgfpoint{31.750000\du}{16.750000\du}}{\pgfpoint{0.750000\du}{0\du}}{\pgfpoint{0\du}{0.250000\du}}
\pgfusepath{stroke}
\definecolor{dialinecolor}{rgb}{1.000000, 1.000000, 1.000000}
\pgfsetfillcolor{dialinecolor}
\pgfpathellipse{\pgfpoint{31.750000\du}{18.250000\du}}{\pgfpoint{0.750000\du}{0\du}}{\pgfpoint{0\du}{0.250000\du}}
\pgfusepath{fill}
\pgfsetlinewidth{0.050000\du}
\pgfsetdash{}{0pt}
\pgfsetdash{}{0pt}
\definecolor{dialinecolor}{rgb}{0.000000, 0.000000, 0.000000}
\pgfsetstrokecolor{dialinecolor}
\pgfpathellipse{\pgfpoint{31.750000\du}{18.250000\du}}{\pgfpoint{0.750000\du}{0\du}}{\pgfpoint{0\du}{0.250000\du}}
\pgfusepath{stroke}
\pgfsetlinewidth{0.050000\du}
\pgfsetdash{}{0pt}
\pgfsetdash{}{0pt}
\pgfsetbuttcap
{
\definecolor{dialinecolor}{rgb}{0.000000, 0.000000, 0.000000}
\pgfsetfillcolor{dialinecolor}
% was here!!!
\definecolor{dialinecolor}{rgb}{0.000000, 0.000000, 0.000000}
\pgfsetstrokecolor{dialinecolor}
\draw (30.023193\du,17.220947\du)--(31.145416\du,16.914886\du);
}
\pgfsetlinewidth{0.050000\du}
\pgfsetdash{}{0pt}
\pgfsetdash{}{0pt}
\pgfsetbuttcap
{
\definecolor{dialinecolor}{rgb}{0.000000, 0.000000, 0.000000}
\pgfsetfillcolor{dialinecolor}
% was here!!!
\definecolor{dialinecolor}{rgb}{0.000000, 0.000000, 0.000000}
\pgfsetstrokecolor{dialinecolor}
\draw (30.023193\du,17.779053\du)--(31.145416\du,18.085114\du);
}
\pgfsetlinewidth{0.050000\du}
\pgfsetdash{}{0pt}
\pgfsetdash{}{0pt}
\pgfsetbuttcap
{
\definecolor{dialinecolor}{rgb}{0.000000, 0.000000, 0.000000}
\pgfsetfillcolor{dialinecolor}
% was here!!!
\definecolor{dialinecolor}{rgb}{0.000000, 0.000000, 0.000000}
\pgfsetstrokecolor{dialinecolor}
\draw (27.975635\du,17.143369\du)--(26.300000\du,16.560000\du);
}
\pgfsetlinewidth{0.050000\du}
\pgfsetdash{}{0pt}
\pgfsetdash{}{0pt}
\pgfsetbuttcap
{
\definecolor{dialinecolor}{rgb}{0.000000, 0.000000, 0.000000}
\pgfsetfillcolor{dialinecolor}
% was here!!!
\definecolor{dialinecolor}{rgb}{0.000000, 0.000000, 0.000000}
\pgfsetstrokecolor{dialinecolor}
\draw (26.300000\du,16.560000\du)--(24.524023\du,15.887666\du);
}
\pgfsetlinewidth{0.050000\du}
\pgfsetdash{}{0pt}
\pgfsetdash{}{0pt}
\pgfsetmiterjoin
\pgfsetbuttcap
\definecolor{dialinecolor}{rgb}{1.000000, 1.000000, 1.000000}
\pgfsetfillcolor{dialinecolor}
\fill (26.300000\du,16.060000\du)--(26.800000\du,16.560000\du)--(26.300000\du,17.060000\du)--(25.800000\du,16.560000\du)--cycle;
\definecolor{dialinecolor}{rgb}{0.000000, 0.000000, 0.000000}
\pgfsetstrokecolor{dialinecolor}
\draw (26.300000\du,16.060000\du)--(26.800000\du,16.560000\du)--(26.300000\du,17.060000\du)--(25.800000\du,16.560000\du)--cycle;
\end{tikzpicture}
}
\end{textblock}
\end{frame}

\section{Machine-Learning-Anwendungsarchitekturen}
\frame{\frametitle{Gliederung} \tableofcontents[currentsection]}
\begin{frame}{Traditionelle Architektur}
\scalebox{\tikzScale}{\adjustTikzSize \input{img/architecture-traditional}}
\end{frame}

\begin{frame}{Datenmodell-getriebene Architektur}
\scalebox{\tikzScale}{\adjustTikzSize \input{img/architecture-dmd}}
\end{frame}

\begin{frame}{Vergleich}
\begin{columns}[t]
	\column{0.5\linewidth}
	\textbf{Traditionelle Architektur}
	
	\vspace*{0.5em}
	\begin{itemize}\setlength{\itemsep}{.45em}
		\item[\pcross] manuelle Datenanbindung an Inferenzalgorithmus
		\item[\pcross] komplexe Modellierungssprache
		\item[\pcross] Inferenzergebnisse nicht einheitlich dokumentiert
		\item[\pcross] Algorithmenauswahl für Laien schwer
		\item[\ptick] lose Kopplung aller Schritte\\
		$\Rightarrow$ kein Wartungsaufwand
	\end{itemize}
	\column{0.5\linewidth}
	\textbf{Datenmodell-getr. Architektur}
	
	\vspace*{0.5em}
	\begin{itemize}\setlength{\itemsep}{.45em}
		\item[\ptick] autom. Datenanbindung nach Modellintegration
		\item[\ptick] einfache Modellierungssprache
		\item[\ptick] Inferenzergebnisse in ERM modelliert
		\item[\ptick] gewählte Algorithmen passend zu Daten
		\item[\pcross] neue Modelle müssen eingepflegt werden
	\end{itemize}
\end{columns}
\end{frame}

\section{Integration probabilistischer und Datenbankmodelle}
\frame{\frametitle{Gliederung} \tableofcontents[currentsection]}
\begin{frame}{Gerichtete graphische Modelle (DGM)}
\textbf{Aufbau}
\begin{itemize}
\item gerichteter Graph
\item Knoten = Zufallsvariablen
\item Kanten = Abhängigkeitsbeziehungen
\item Abwesenheit von Kanten = Unabhängigkeitsannahmen
\item beobachtete Variablen = grau schattiert,\\Hyperparameter = schwarz ausgefüllt
\end{itemize}

\textbf{Ziele}
\begin{itemize}
\item graphische Modellierung von Unabhängigkeitsannahmen\\$\Rightarrow$ Faktorisierung der Verbundverteilung
\item Schlüsse über (bedingte) Unabhängigkeitsbeziehungen auf Basis der Graphstruktur ziehen: \emph{d-Separation}\scite{pearl1988probabilistic}
\end{itemize}

\begin{textblock}{10}(85,18)
	\scalebox{\tikzScale}{\adjustTikzSize % Graphic for TeX using PGF
% Title: /Users/frankrosner/Documents/Uni/MasterThesis/presentation/img/simple-dgm.dia
% Creator: Dia v0.97.2
% CreationDate: Sun Sep 21 12:38:09 2014
% For: frankrosner
% \usepackage{tikz}
% The following commands are not supported in PSTricks at present
% We define them conditionally, so when they are implemented,
% this pgf file will use them.
\ifx\du\undefined
  \newlength{\du}
\fi
\setlength{\du}{15\unitlength}
\begin{tikzpicture}
\pgftransformxscale{1.000000}
\pgftransformyscale{-1.000000}
\definecolor{dialinecolor}{rgb}{0.000000, 0.000000, 0.000000}
\pgfsetstrokecolor{dialinecolor}
\definecolor{dialinecolor}{rgb}{1.000000, 1.000000, 1.000000}
\pgfsetfillcolor{dialinecolor}
\definecolor{dialinecolor}{rgb}{1.000000, 1.000000, 1.000000}
\pgfsetfillcolor{dialinecolor}
\pgfpathellipse{\pgfpoint{16.500000\du}{11.500000\du}}{\pgfpoint{0.500000\du}{0\du}}{\pgfpoint{0\du}{0.500000\du}}
\pgfusepath{fill}
\pgfsetlinewidth{0.100000\du}
\pgfsetdash{}{0pt}
\pgfsetdash{}{0pt}
\definecolor{dialinecolor}{rgb}{0.000000, 0.000000, 0.000000}
\pgfsetstrokecolor{dialinecolor}
\pgfpathellipse{\pgfpoint{16.500000\du}{11.500000\du}}{\pgfpoint{0.500000\du}{0\du}}{\pgfpoint{0\du}{0.500000\du}}
\pgfusepath{stroke}
% setfont left to latex
\definecolor{dialinecolor}{rgb}{0.000000, 0.000000, 0.000000}
\pgfsetstrokecolor{dialinecolor}
\node[anchor=west] at (15.881100\du,10.410800\du){$x$};
\definecolor{dialinecolor}{rgb}{1.000000, 1.000000, 1.000000}
\pgfsetfillcolor{dialinecolor}
\pgfpathellipse{\pgfpoint{12.000000\du}{11.500000\du}}{\pgfpoint{0.500000\du}{0\du}}{\pgfpoint{0\du}{0.500000\du}}
\pgfusepath{fill}
\pgfsetlinewidth{0.100000\du}
\pgfsetdash{}{0pt}
\pgfsetdash{}{0pt}
\definecolor{dialinecolor}{rgb}{0.000000, 0.000000, 0.000000}
\pgfsetstrokecolor{dialinecolor}
\pgfpathellipse{\pgfpoint{12.000000\du}{11.500000\du}}{\pgfpoint{0.500000\du}{0\du}}{\pgfpoint{0\du}{0.500000\du}}
\pgfusepath{stroke}
% setfont left to latex
\definecolor{dialinecolor}{rgb}{0.000000, 0.000000, 0.000000}
\pgfsetstrokecolor{dialinecolor}
\node[anchor=west] at (11.381100\du,10.3500000\du){$\theta$};
\pgfsetlinewidth{0.100000\du}
\pgfsetdash{}{0pt}
\pgfsetdash{}{0pt}
\pgfsetbuttcap
{
\definecolor{dialinecolor}{rgb}{0.000000, 0.000000, 0.000000}
\pgfsetfillcolor{dialinecolor}
% was here!!!
\pgfsetarrowsend{to}
\definecolor{dialinecolor}{rgb}{0.000000, 0.000000, 0.000000}
\pgfsetstrokecolor{dialinecolor}
\draw (12.500000\du,11.500000\du)--(16.000000\du,11.500000\du);
}
% setfont left to latex
\definecolor{dialinecolor}{rgb}{0.000000, 0.000000, 0.000000}
\pgfsetstrokecolor{dialinecolor}
\node[anchor=west] at (6.875000\du,13.500000\du){$x \sim Bern(\theta)$};
\definecolor{dialinecolor}{rgb}{0.000000, 0.000000, 0.000000}
\pgfsetfillcolor{dialinecolor}
\pgfpathellipse{\pgfpoint{7.500000\du}{11.500000\du}}{\pgfpoint{0.300000\du}{0\du}}{\pgfpoint{0\du}{0.300000\du}}
\pgfusepath{fill}
\pgfsetlinewidth{0.100000\du}
\pgfsetdash{}{0pt}
\pgfsetdash{}{0pt}
\definecolor{dialinecolor}{rgb}{0.000000, 0.000000, 0.000000}
\pgfsetstrokecolor{dialinecolor}
\pgfpathellipse{\pgfpoint{7.500000\du}{11.500000\du}}{\pgfpoint{0.300000\du}{0\du}}{\pgfpoint{0\du}{0.300000\du}}
\pgfusepath{stroke}
% setfont left to latex
\definecolor{dialinecolor}{rgb}{0.000000, 0.000000, 0.000000}
\pgfsetstrokecolor{dialinecolor}
\node[anchor=west] at (6.881100\du,10.500000\du){$\alpha$};
\pgfsetlinewidth{0.100000\du}
\pgfsetdash{}{0pt}
\pgfsetdash{}{0pt}
\pgfsetbuttcap
{
\definecolor{dialinecolor}{rgb}{0.000000, 0.000000, 0.000000}
\pgfsetfillcolor{dialinecolor}
% was here!!!
\pgfsetarrowsend{to}
\definecolor{dialinecolor}{rgb}{0.000000, 0.000000, 0.000000}
\pgfsetstrokecolor{dialinecolor}
\draw (7.800000\du,11.500000\du)--(11.500000\du,11.500000\du);
}
% setfont left to latex
\definecolor{dialinecolor}{rgb}{0.000000, 0.000000, 0.000000}
\pgfsetstrokecolor{dialinecolor}
\node[anchor=west] at (6.875000\du,14.950000\du){$\theta \sim Beta(\alpha)$};
% setfont left to latex
\definecolor{dialinecolor}{rgb}{0.000000, 0.000000, 0.000000}
\pgfsetstrokecolor{dialinecolor}
\node[anchor=west] at (6.875000\du,16.400000\du){$P(\theta, x | \alpha) = P(x | \theta) \cdot P(\theta | \alpha)$};
\end{tikzpicture}
}
\end{textblock}
\end{frame}

\begin{frame}{Plate-Modelle}
\textbf{Motivation}
\begin{itemize}
\item DGMs nur für flache Daten geeignet
\item hierarchische Daten erfordern wiederholte Variablen:\\
z.B. Münzenwürfe $x_m$ unterschiedlicher Münzen mit eigenem Parameter $\theta_m$
\item jede Variable einzeln zu zeichnen führt zu hohem Platzbedarf und ist schnell unübersichtlich
\end{itemize}

\vspace*{0.5em}
\textbf{Plate-Modell}
\begin{itemize}
\item zeichne wiederholte Variablen\\ in einen Kasten (sog. Plate)
\item ordne jeder Plate eine Indexmenge zu
\item indiziere alle Variablen innerhalb der Plate über diese Menge
\end{itemize}

\begin{textblock}{10}(80,60)
	\scalebox{\tikzScale}{\adjustTikzSize % Graphic for TeX using PGF
% Title: /Users/frankrosner/Documents/Uni/MasterThesis/presentation/img/simple-pm.dia
% Creator: Dia v0.97.2
% CreationDate: Sun Sep 21 12:36:27 2014
% For: frankrosner
% \usepackage{tikz}
% The following commands are not supported in PSTricks at present
% We define them conditionally, so when they are implemented,
% this pgf file will use them.
\ifx\du\undefined
  \newlength{\du}
\fi
\setlength{\du}{15\unitlength}
\begin{tikzpicture}
\pgftransformxscale{1.000000}
\pgftransformyscale{-1.000000}
\definecolor{dialinecolor}{rgb}{0.000000, 0.000000, 0.000000}
\pgfsetstrokecolor{dialinecolor}
\definecolor{dialinecolor}{rgb}{1.000000, 1.000000, 1.000000}
\pgfsetfillcolor{dialinecolor}
\pgfsetlinewidth{0.100000\du}
\pgfsetdash{}{0pt}
\pgfsetdash{}{0pt}
\pgfsetmiterjoin
\definecolor{dialinecolor}{rgb}{1.000000, 1.000000, 1.000000}
\pgfsetfillcolor{dialinecolor}
\fill (14.200000\du,3.500000\du)--(14.200000\du,9.000000\du)--(23.000000\du,9.000000\du)--(23.000000\du,3.500000\du)--cycle;
\definecolor{dialinecolor}{rgb}{0.000000, 0.000000, 0.000000}
\pgfsetstrokecolor{dialinecolor}
\draw (14.200000\du,3.500000\du)--(14.200000\du,9.000000\du)--(23.000000\du,9.000000\du)--(23.000000\du,3.500000\du)--cycle;
\definecolor{dialinecolor}{rgb}{1.000000, 1.000000, 1.000000}
\pgfsetfillcolor{dialinecolor}
\pgfpathellipse{\pgfpoint{21.000000\du}{6.500000\du}}{\pgfpoint{0.500000\du}{0\du}}{\pgfpoint{0\du}{0.500000\du}}
\pgfusepath{fill}
\pgfsetlinewidth{0.100000\du}
\pgfsetdash{}{0pt}
\pgfsetdash{}{0pt}
\definecolor{dialinecolor}{rgb}{0.000000, 0.000000, 0.000000}
\pgfsetstrokecolor{dialinecolor}
\pgfpathellipse{\pgfpoint{21.000000\du}{6.500000\du}}{\pgfpoint{0.500000\du}{0\du}}{\pgfpoint{0\du}{0.500000\du}}
\pgfusepath{stroke}
% setfont left to latex
\definecolor{dialinecolor}{rgb}{0.000000, 0.000000, 0.000000}
\pgfsetstrokecolor{dialinecolor}
\node[anchor=west] at (20.381100\du,5.450800\du){$x_m$};
\definecolor{dialinecolor}{rgb}{1.000000, 1.000000, 1.000000}
\pgfsetfillcolor{dialinecolor}
\pgfpathellipse{\pgfpoint{16.500000\du}{6.500000\du}}{\pgfpoint{0.500000\du}{0\du}}{\pgfpoint{0\du}{0.500000\du}}
\pgfusepath{fill}
\pgfsetlinewidth{0.100000\du}
\pgfsetdash{}{0pt}
\pgfsetdash{}{0pt}
\definecolor{dialinecolor}{rgb}{0.000000, 0.000000, 0.000000}
\pgfsetstrokecolor{dialinecolor}
\pgfpathellipse{\pgfpoint{16.500000\du}{6.500000\du}}{\pgfpoint{0.500000\du}{0\du}}{\pgfpoint{0\du}{0.500000\du}}
\pgfusepath{stroke}
% setfont left to latex
\definecolor{dialinecolor}{rgb}{0.000000, 0.000000, 0.000000}
\pgfsetstrokecolor{dialinecolor}
\node[anchor=west] at (15.881100\du,5.3800000\du){$\theta_m$};
\pgfsetlinewidth{0.100000\du}
\pgfsetdash{}{0pt}
\pgfsetdash{}{0pt}
\pgfsetbuttcap
{
\definecolor{dialinecolor}{rgb}{0.000000, 0.000000, 0.000000}
\pgfsetfillcolor{dialinecolor}
% was here!!!
\pgfsetarrowsend{to}
\definecolor{dialinecolor}{rgb}{0.000000, 0.000000, 0.000000}
\pgfsetstrokecolor{dialinecolor}
\draw (17.000000\du,6.500000\du)--(20.500000\du,6.500000\du);
}
\definecolor{dialinecolor}{rgb}{0.000000, 0.000000, 0.000000}
\pgfsetfillcolor{dialinecolor}
\pgfpathellipse{\pgfpoint{12.000000\du}{6.500000\du}}{\pgfpoint{0.300000\du}{0\du}}{\pgfpoint{0\du}{0.300000\du}}
\pgfusepath{fill}
\pgfsetlinewidth{0.100000\du}
\pgfsetdash{}{0pt}
\pgfsetdash{}{0pt}
\definecolor{dialinecolor}{rgb}{0.000000, 0.000000, 0.000000}
\pgfsetstrokecolor{dialinecolor}
\pgfpathellipse{\pgfpoint{12.000000\du}{6.500000\du}}{\pgfpoint{0.300000\du}{0\du}}{\pgfpoint{0\du}{0.300000\du}}
\pgfusepath{stroke}
% setfont left to latex
\definecolor{dialinecolor}{rgb}{0.000000, 0.000000, 0.000000}
\pgfsetstrokecolor{dialinecolor}
\node[anchor=west] at (11.381100\du,5.500000\du){$\alpha$};
\pgfsetlinewidth{0.100000\du}
\pgfsetdash{}{0pt}
\pgfsetdash{}{0pt}
\pgfsetbuttcap
{
\definecolor{dialinecolor}{rgb}{0.000000, 0.000000, 0.000000}
\pgfsetfillcolor{dialinecolor}
% was here!!!
\pgfsetarrowsend{to}
\definecolor{dialinecolor}{rgb}{0.000000, 0.000000, 0.000000}
\pgfsetstrokecolor{dialinecolor}
\draw (12.300000\du,6.500000\du)--(16.000000\du,6.500000\du);
}
% setfont left to latex
\definecolor{dialinecolor}{rgb}{0.000000, 0.000000, 0.000000}
\pgfsetstrokecolor{dialinecolor}
\node[anchor=east] at (23.075000\du,8.375000\du){$m \in M$};
\end{tikzpicture}
}
\end{textblock}
\end{frame}

\section{Schlussbetrachtung}
\frame{\frametitle{Gliederung} \tableofcontents[currentsection]}
\begin{frame}

\end{frame}

\section*{Quellen}
\frame{\frametitle{Gliederung} \tableofcontents[currentsection]}
\begin{frame}
\frametitle{Quellen}
\def\bibfont{\scriptsize}
\printbibliography
\end{frame}

\begin{frame}{Test}

x
\begin{definition}[Definition]
y
\end{definition}

\begin{bem}[Bemsss]
x
\end{bem}
\end{frame}

\end{document}