\section{Related Work}

Providing easy-to-use interfaces for machine learning algorithms is highly topical. The MLbase team of the University of California, Berkeley, currently works on an application programming interface (API) to build distributed machine learning algorithm implementations with minimal complexity and competitive performance and scalability (\cite{sparks2013mli}). They focus on closing the gap between an algorithm prototype written by a scientist and the final product created by software engineers. 

\textcite{akdere2011case} propose that next generation database management systems should incorporate machine learning techniques to allow data-driven predictive analysis. They developed a first prototype, named Longview, of a so called \emph{predictive database management system} (PDBMS), which aims to natively support and manage probabilistic models. Those models are accessed by calling structured query language (SQL) functions or using special relations. MADlib (\cite{hellerstein2012madlib}) is an analytics library for relational database systems built on SQL statements. Although these approaches are similar to our data model driven one, they operate on SQL and thus are not as generic and easy to understand as ERMs. Additionally, they do not provide an intuitive way to select inference algorithms that integrate well with the available data.

The Hazy project (\cite{kumar2013hazy}) provides programming and infrastructure abstractions to enable users to quickly build new systems. Statistical processing abstractions are based on Markov logic (\cite{domingos2007markov}). The goal is to integrate these techniques with data processing systems. Programming, infrastructure, and statistical processing abstractions allow to reuse existing code when building custom applications, but they require a deeper understanding of the algorithms in order to combine them effectively.

Traditional machine learning algorithm libraries like Weka (\cite{hall2009weka}) and scikit-learn (\cite{scikit-learn}) enable users to quickly select and try different algorithms. While Weka has a graphical user interface (GUI), the algorithms in scikit-learn are accessed using the Python programming language. The problem is that both programs do not offer an easy way to integrate the results with domain specific meta data, and have only textual explanations of what the algorithms do. Our data model driven approach allows better data integration and visualizes how the inference results look like.

\newpage

Table~\ref{tab:related_work} compares all projects mentioned above. For each project, we specify the given user interface, whether it is suitable for a technical and / or a non-technical audience, and if it offers an intuitive and easy way to integrate meta data. While all applications are suitable for a technical audience by having a flexible API, only Weka, and related projects which include a GUI, allow non-technical users to apply machine learning techniques. Our data-model driven framework is flexible enough for technical users, provides a comprehensible user interface via ERMs for a non-technical audience, and allows easy meta data integration by matching two ERMs.

\begin{table}[t]
\centering
\begin{tabular}{llccc} %\parbox[t]{3cm}{A\\B} for line breaks in tabulars
& & \multicolumn{2}{c}{\textbf{Target Audience:}} & \textbf{Easy Data}\\
\textbf{Name} & \textbf{Interface} & \textbf{Techn.}     & \textbf{Non-techn.} & \textbf{Integration?}\\
\hline
MLbase & Scala & 		\checkmark & $\times$ & $\times$\\
Longview & SQL &		\checkmark & $\times$ & $\times$\\
MADlib & SQL & 		\checkmark & $\times$ & $\times$\\
Hazy & Markov logic & 			\checkmark & $\times$ & $\times$\\
Weka & Java / GUI & 			\checkmark & \checkmark & $\times$\\
Scikit & Python & 		\checkmark & $\times$ & $\times$\\
\hdashline
Our approach & ERM / GUI &	\checkmark & \checkmark & \checkmark
\end{tabular}
\caption[Comparison of related projects]{Comparison of related projects. We present the given user interface, whether it is suitable for a technical and / or a non-technical audience, and if it offers an intuitive and easy way to integrate meta data.}\label{tab:related_work}
\end{table}
