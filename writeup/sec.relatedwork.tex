\section{Related Work}

Providing easy-to-use interfaces for machine learning algorithms is highly topical. The MLbase team of the University of California, Berkeley currently works on an application programming interface (API) to build distributed machine learning algorithm implementations with minimal complexity and competitive performance and scalability (\cite{sparks2013mli}). They focus on closing the gap between an algorithm prototype written by a scientist and the final product created by software engineers.

\textcite{akdere2011case} propose that next generation database management systems should incorporate machine learning techniques to allow data-driven predictive analysis. They developed a first prototype, named Longview, of a so called \emph{predictive database management system} (PDBMS), which aims to natively support and manage probabilistic models. Those models are accessed by calling structured query language (SQL) functions or using special relations. MADlib (\cite{hellerstein2012madlib}) is an analytics library for relational database systems built on SQL statements. Although those approaches are similar to our data model driven one, they operates on SQL and thus is not as generic and easy to understand as ERMs. Additionally, it does not provide an intuitive way to select inference algorithms that integrate well with the available data.

The Hazy project (\cite{kumar2013hazy}) tries to provide programming and infrastructure abstractions to enable a user to quickly build new systems. Statistical processing abstractions are based on Markov logic (\cite{domingos2007markov}). The goal is to integrate these techniques with data processing systems.

Traditional machine learning algorithm libraries like Weka (\cite{hall2009weka}) and scikit-learn (\cite{scikit-learn}) use a graphical user interface to enable users to quickly select and try different algorithms. The problem is that they do not offer an easy way to integrate the results with domain specific meta data and offer only a textual explanation of what the algorithms do. Our data model driven approach allows better data integration and visualizes how the inference results look like.

[TODO: Add BUGS and Infer.NET]

All these approaches do not provide an easy way to select and understand inference algorithms and to integrate existing meta data into the analysis like our approach does, without requiring to have a machine learning expert at hand. [TODO: expand section a lil bit, rethink this paragraph]

\begin{table}[t]
\centering
\begin{tabular}{llccc} %\parbox[t]{3cm}{A\\B} for line breaks in tabulars
& & \multicolumn{2}{c}{\textbf{Target Audience:}} & \textbf{Easy Data}\\
\textbf{Name} & \textbf{Interface} & \textbf{Techn.}     & \textbf{Non-techn.} & \textbf{Integration?}\\
\hline
MLbase & Scala & 		\checkmark & $\times$ & $\times$\\
Longview & SQL &		\checkmark & $\times$ & $\times$\\
MADlib & SQL & 		\checkmark & $\times$ & $\times$\\
Hazy & Markov logic & 			\checkmark & $\times$ & $\times$\\
Weka & Java / GUI & 			\checkmark & \checkmark & $\times$\\
Scikit & Python & 		\checkmark & $\times$ & $\times$\\
Our approach & ERM / GUI &	\checkmark & \checkmark & \checkmark
\end{tabular}
\caption[Comparison of related projects]{Comparison of related projects. [TODO: Description]}\label{tab:related_work}
\end{table}
